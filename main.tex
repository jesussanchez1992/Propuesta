\documentclass[11pt]{article}

\usepackage{amsmath}
\usepackage{amssymb}
\usepackage[utf8]{inputenc}
%\usepackage[latin1]{inputenc}
\usepackage[spanish]{babel}
\usepackage[left=3cm,right=3cm,top=3cm,bottom=2.5cm]{geometry}
\usepackage{amsmath,amssymb,latexsym,color,graphicx,verbatim}
\usepackage{mathrsfs}
\usepackage{layout}
\usepackage{graphicx}

%COLOCA COMANDOS EN ESPAÑOL
%\renewcommand{\contentsname}{Contenido}
%\renewcommand{\partname}{Parte}
%\renewcommand{\appendixname}{Apéndice}
%\renewcommand{\figurename}{Figura}
%\renewcommand{\tablename}{Tabla}
%\renewcommand{\abstractname}{Resumen}
%\renewcommand{\refname}{Bibliografía}
%FIN DEL BLOQUE
\usepackage[acronym,shortcuts]{glossaries} %PARA UN GLOSARIO DE ACRÓNIMOS
\makeglossaries
\usepackage[font=small]{caption}
\usepackage[colorlinks = true,
                     linkcolor = blue,
                     citecolor = red,
                     urlcolor = blue]{hyperref}

\baselineskip0.75cm
\parskip0.5cm
\parindent0cm

\begin{document}


\begin{titlepage}
\centering {\Large {\sc Calibración del espectrografo EShel- II PF0011}}

\vfill
\centering {\Large Propuesta de trabajo de grado para optar al t\'itulo de F\'isico}
\vfill
\hfill

\centering {\Large Jesús Alberto Sánchez Villafrades$^{1,2}$}

\vfill

\centering {\Large Director: Luis Alberto Núñez$^{1,2}$}

\centering {\Large Co-Director: $^{3}$}



\hfill






{{\Large $^1$Grupo de Investigaci\'on en Relatividad y Gravitaci\'on GIRG}} \\
{{\Large$^2$Grupo Halley de Astronom\'ia y Ciencias Aeroespaciales}} \\

\vfill
\vfill

\centering {\Large Universidad Industrial de Santander\\Facultad de
Ciencias\\Escuela de F\'{i}sica\\Bucaramanga\\2018}


\end{titlepage}


\newpage

\tableofcontents

%\newpage

%\acrodef{LAGO}{Latin American Giant Observatory}

\newpage


\begin{abstract}



 

%Surge entonces, la necesidad de mejorar las capacidades de la red de detección. En este trabajo se propone estudiar la pertinencia de la construcción y uso de perfiles atmosféricos mensuales, basados en datos extraídos del Global Data Assimilation System  (GDAS) para los sitios de LAGO. Así mismo, para verificar la efectividad de esta implementación los cálculos del flujo serán contrastados entre si.

\vspace{0.5cm}
Palabras clave: 

\end{abstract}


%%%%%%%%%%%%%%%%%%%%%%%%%%%%%%%%%%%%%%%%%%%%%%%%%%%%%%%%%%%%%%%%%%%%%%%%%%%%%%%%%%%%%%%%%%%%%%
\section{Introducci\'on}

Todo en el mundo se hace con dinero y otras cosas mas y mas y mas
%%%%%%%%%%%%%%%%%%%%%%%%%%%%%%%%%%%%%%%%%%%%%%%%%%%%%%%%%%%%%%%%%%%%%%%%%%%%%%%%%%%%%%%%%%%%%%








%%%%%%%%%%%%%%%%%%%%%%%%%%%%%%%%%%%%%%%%%%%%%%%%%%%%%%%%%%%%




\section{Marco teórico}
La espectroscopia estudia la interacción entre la radiación electromagnética y la metería,en astronomía esta radiación es emitida por estrellas y otros objetos celestes y al ser dispersada brinda mucha información de la fuente que la genero.Para el caso del espectro visible esta Luz puede ser dispersada usando un prisma mediante el fenómeno de refracción o usando una rejilla de difracción, estos espectros dan información importante de las características físicas del objeto que las emite,están directamente relacionadas con la temperatura superficial del objeto así como con su composición química. Usando Efecto Dopler se puede tener información de la velocidad de rotación y traslación además de densidad y presión.
\cite{moreira2009modelo}\\


%%%%%%%%%%%%%%%%%%%%%%%%%%%%%%%%%%%%%%%%%%%%%%%%%%%%%%%%%%%%%%%%%%%%%%%%%%%%%%%%%%%%%%%%%%%%%%

\subsection {Principios de la espectroscopia.}



%%%%%%%%%%%%%%%%%%%%%%%%%%%%%%%%%%%%%%%



%%%%%%%%%%%%%%%%%%%%%%%%%%%%%%%%%%%%%%%%%%%%%%%%%%%%%%%%%%%%%%%%%%%%%%%%%%%%%%%%%%%%%%%%%%%%%%
\section{Objetivos}

%%%%%%%%%%%%%%%%%%%%%%%%%%%%%%%%%%%%%%%%%%%%%%%%%%%%%%%%%%%%%%%%%%%%%%%%%%%%%%%%%%%%%%%%%%%%%

\begin{itemize}
\item \textbf{Objetivo General}

Realizar el montaje, calibración y puesta en funcionamiento de ESPECTROGRAFO eShel II con el que cuenta el Grupo Halley de Astronoḿıa y Ciencias Aeroespacialesde la Universidad Industrial de Santander.\\

\item \textbf{Objetivos Espec\'ificos}
\end{itemize}

\begin{itemize}

\item Calcular de forma experimental en ancho de la rendija micrométrica que permite el paso de luz el espectrógrafo.

\item Verificar el enfoque del lente colimador del espectrógrafo eShel II verificándolos con lineas de lamparas de emisión.

\item Realizar la calibración pixel-Longitud de onda mediante el software IRAF, usando lamparas de calibración con espectros conocidos.

\item Realizar los cálculos de la masa de aire para la ciudad de Bucaramanga de forma teórica.

\end{itemize}
%%%%%%%%%%%%%%%%%%%%%%%%%%%%%%%%%%%%%%%%%%%%%%%%%%%%%%%%%%%%%%%%%%%%%%%%%%%%%%%%%%%%%%%%%%%%%%




\newpage

%%%%%%%%%%%%%%%%%%%%%%%%%%%%%%%%%%%%%%%%%%%%%%%%%%%%%%%%%%%%%%%%%%%%%%%%%%%%%%%%%%%%%%%%%%%%%%
%%%%%%%%%%%%%%%%%%%%%%%%%%%%%%%%%%%%%%%%%%%%%%%%%%%%%%%%%%%%%%%%%%%%%%%%%%%%%%%%%%%%%%%%%%%%%%
\section{Metodolog\'ia}

Para cumplir con los objetivos planteados se llevara la siguiente metodología.

\begin{itemize}

\item[1] Montaje del espectrógrafo con sus respectivo modulo de calibración y acople al telescopio.

\item[2] hacer el calculo de forma experimental de la rendija del espejo que permite el paso de luz al espectrógrafo.

\item[3] Realizar la captura de imágenes limpias de diferentes lamparas de emisión de laboratorio con el fin de garantizar que los elementos dispersores del espectrógrafo estén alineados con la cámara y se puedan reproducir espectros conocidos.

\item[3] Se calibrara la Montura robotizada PARAMOUNT  usando el software T-Point para garantizar un correcto apunte del telescopio al objeto de interés.



 
\end{itemize}

%%%%%%%%%%%%%%%%%%%%%%%%%%%%%%%%%%%%%%%%%%%%%%%%%%%%%%%%%%%%%%%%%%%%%%%%%%%%%%%%%%%%%%%%%%%%%%


%%%%%%%%%%%%%%%%%%%%%%%%%%%%%%%%%%%%%%%%%%%%%%%%%%%%%%%%%%%%%%%%%%%%%%%%%%%%%%%%%%%%%%%%%%%%%%
\section{Cronograma de Actividades}	
%%%%%%%%%%%%%%%%%%%%%%%%%%%%%%%%%%%%%%%%%%%%%%%%%%%%%%%%%%%%%%%%%%%%%%%%%%%%%%%%%%%%%%%%%%%%%%


\begin{center}
{\small
\begin{tabular}{|c|c|c|c|c|c|c|c|}
\hline

\textbf{Mes}/\textbf{Actividad}&\textbf{Act 1.1}&\textbf{Act 1.2}
&\textbf{Act 1.3}&\textbf{Act 2}&\textbf{Act 3}&\textbf{Act 4}&\textbf{Act 5}\\

\hline

Enero&$\bigotimes$&&&&&&\\

\hline

Febrero&$\bigotimes$&$\bigotimes$&&&&&\\

\hline

Marzo&&$\bigotimes$&$\bigotimes$&&&&\\

\hline

Abril&&&$\bigotimes$&&&&\\

\hline

Mayo&&&$\bigotimes$&$\bigotimes$&&&\\

\hline

Junio&&&$\bigotimes$&$\bigotimes$&&&\\

\hline
Julio&&&$\bigotimes$&$\bigotimes$&$\bigotimes$&&$\bigotimes$\\

\hline

Agosto&&&&&$\bigotimes$&$\bigotimes$&\\

\hline 

Septiembre&&&&&&&$\bigotimes$\\

\hline
Octubre&&&&&&&$\bigotimes$\\
\hline

\end{tabular}
}
\end{center}



%%%%%%%%%%%%%%%%%%%%%%%%%%%%%%%%%%%%%%%%%%%%%%%%%%%%%%%%%%%%%%%%%%%%%%%%%%%%%%%%%%%%%%%%%%%%%%
\medskip
\bibliographystyle{unsrt}
\bibliography{moreira.bib}





%%%%%%%%%%%%%%%%%%%%%%%%%%%%%%%%%%%%%%%%%%%%%%%%%%%%%%%%%%%%%%%%%%%%%%%%%%%%%%%%%%%%%%%%%%%%%%
\end{document}
